\subsection*{Constructing the DFA}
We obtain the DFA by the subset construction method. The states are

\begin{equation}
\begin{split}
    & c_1 = \{q_1, q_2\},\\
    & c_2 = \{q_1, q_2, q_3\},\\
    & c_3 = \{q_1, q_2, q_4\}\\
    & c_4 = \emptyset \\
\end{split}
\end{equation}

\[ 
D_1 = \big( \{c_1, c_2, c_3, c_4\}, \{a,b\}, \delta, c_1, \{c_1, c_2, c_3\} \big)
\]

where $\delta$ is defined as:
\begin{tabular}{|l|l|l|}
\hline
    & a   & b   \\ \hline
$c_1$ & $c_2$ & $c_4$ \\
$c_2$ & $c_2$ & $c_3$ \\
$c_3$ & $c_2$ & $c_4$ \\
$c_4$ & $c_4$ & $c_4$ \\ \hline
\end{tabular}

\begin{figure}[ht]
\centering
\begin{tikzpicture}
    \node[state, accepting, initial] (q0) {$c_1$};
    \node[state, accepting, right of=q0] (q1) {$c_2$};
    \node[state, accepting, right of=q1] (q2) {$c_3$};
    \node[state, below of=q1] (q3) {$c_4$};
    
    \draw   (q0) edge[above] node{$a$} (q1)
            (q0) edge[below] node{$b$} (q3)
            (q1) edge[loop above] node{$a$} (q1)
            (q1) edge[above, bend left] node {$b$} (q2)
            (q2) edge[below, bend left] node {$a$} (q1)
            (q2) edge[below] node {$b$} (q3)
            (q3) edge[loop below] node{$a,b$} (q3);
            
\end{tikzpicture}
\caption{DFA $D_1$}
\label{fig:dfa}
\end{figure}

% ----------------------------------------------------------------


We can find the complement of the DFA drawn above by flipping the
accepting to non-accepting states and vice-versa. This gives
us the DFA in~Fig~\ref{fig:comp_dfa}

\[ 
D_2 = \big( \{d_1, d_2, d_3, d_4\}, \{a,b\}, \delta, d_1, \{d_4\} \big) \]

where $\delta$ is defined as:
\begin{tabular}{|l|l|l|}
\hline
    & a   & b   \\ \hline
$d_1$ & $d_2$ & $d_4$ \\
$d_2$ & $d_2$ & $d_3$ \\
$d_3$ & $d_2$ & $d_4$ \\
$d_4$ & $d_4$ & $d_4$ \\ \hline
\end{tabular}
\begin{figure}[H] 
\centering
\begin{tikzpicture}
    \node[state, initial] (q0) {$d_1$};
    \node[state, right of=q0] (q1) {$d_2$};
    \node[state, right of=q1] (q2) {$d_3$};
    \node[state, accepting, below of=q1] (q3) {$d_4$};
    
    \draw   (q0) edge[above] node{$a$} (q1)
            (q0) edge[below] node{$b$} (q3)
            (q1) edge[loop above] node{$a$} (q1)
            (q1) edge[above, bend left] node {$b$} (q2)
            (q2) edge[below, bend left] node {$a$} (q1)
            (q2) edge[below] node {$b$} (q3)
            (q3) edge[loop below] node{$a,b$} (q3);
            
\end{tikzpicture}
\caption{Complemented DFA $D_2$}
\label{fig:comp_dfa}
\end{figure}
