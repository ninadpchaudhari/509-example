\subsection*{Obtaining the NFA}
An NFA for the given example is:

$$
N_1 = (\{q_1,q_2, q_3, q_4\}, \{a,b\}, \delta, q_1, \{q_1\})
$$
where $\delta$ is defined as : 

\begin{tabular}{|l|l|l|l|}
\hline
     & $\varepsilon$ & a        & b        \\\hline
$q_1$ & $\{q_2\}$                   &          &          \\
$q_2$ &                            & $\{q_3\}$ &          \\
$q_3$ & $\{q_1\}$                   &          & $\{q_4\}$ \\
$q_4$ & $\{q_1\}$                   &          &        \\ \hline
\end{tabular}

\begin{figure}[ht]
\centering
\begin{tikzpicture}
    \node[state, accepting, initial] (q0) {$q_1$};
    \node[state, right of=q0] (q1) {$q_2$};
    \node[state, right of=q1] (q2) {$q_3$};
    \node[state, right of=q2] (q3) {$q_4$};
    
    \draw   (q0) edge[above] node{$\varepsilon$} (q1)
            (q1) edge[above] node{$a$} (q2)
            (q2) edge[above] node{$b$} (q3)
            (q2) edge[above, bend right] node{$\varepsilon$} (q0)
            (q3) edge[above, bend left] node{$\varepsilon$} (q0);
            
\end{tikzpicture}
\caption{NFA for $R$}
\label{fig:nfa}
\end{figure}

Now observe the $\mathcal{E}$ closures of all states in fig~\ref{fig:nfa}:
\begin{equation}
\label{eq:closure}
\begin{split}
&\mathcal{E} (q_1) = \{q_1, q_2\} \\
&\mathcal{E} (q_2) = \{q_2\} \\
&\mathcal{E} (q_3) = \{q_1, q_2, q_3\} \\
&\mathcal{E} (q_4) = \{q_1, q_2, q_4\} \\
\end{split}
\end{equation}
